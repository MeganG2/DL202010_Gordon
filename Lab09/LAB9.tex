% Digital Logic Lab 9 ALU with Input Register
% Created: 2020-04-02, Megan Gordon

%==========================================================
%=========== Document Setup  ==============================

% Formatting defined by class file
\documentclass[11pt]{article}

% ---- Document formatting ----
\usepackage[margin=1in]{geometry}	% Narrower margins
\usepackage{booktabs}				% Nice formatting of tables
\usepackage{graphicx}				% Ability to include graphics

%\setlength\parindent{0pt}	% Do not indent first line of paragraphs 
\usepackage[parfill]{parskip}		% Line space b/w paragraphs
%	parfill option prevents last line of pgrph from being fully justified

% Parskip package adds too much space around titles, fix with this
\RequirePackage{titlesec}
\titlespacing\section{0pt}{8pt plus 4pt minus 2pt}{3pt plus 2pt minus 2pt}
\titlespacing\subsection{0pt}{4pt plus 4pt minus 2pt}{-2pt plus 2pt minus 2pt}
\titlespacing\subsubsection{0pt}{2pt plus 4pt minus 2pt}{-6pt plus 2pt minus 2pt}

% ---- Hyperlinks ----
\usepackage[colorlinks=true,urlcolor=blue]{hyperref}	% For URL's. Automatically links internal references.

% ---- Code listings ----
\usepackage{listings} 					% Nice code layout and inclusion
\usepackage[usenames,dvipsnames]{xcolor}	% Colors (needs to be defined before using colors)

% Define custom colors for listings
\definecolor{listinggray}{gray}{0.98}		% Listings background color
\definecolor{rulegray}{gray}{0.7}			% Listings rule/frame color

% Style for Verilog
\lstdefinestyle{Verilog}{
	language=Verilog,					% Verilog
	backgroundcolor=\color{listinggray},	% light gray background
	rulecolor=\color{blue}, 			% blue frame lines
	frame=tb,							% lines above & below
	linewidth=\columnwidth, 			% set line width
	basicstyle=\small\ttfamily,	% basic font style that is used for the code	
	breaklines=true, 					% allow breaking across columns/pages
	tabsize=3,							% set tab size
	commentstyle=\color{gray},	% comments in italic 
	stringstyle=\upshape,				% strings are printed in normal font
	showspaces=false,					% don't underscore spaces
}

% How to use: \Verilog[listing_options]{file}
\newcommand{\Verilog}[2][]{%
	\lstinputlisting[style=Verilog,#1]{#2}
}




%======================================================
%=========== Body  ====================================
\begin{document}

\title{ELC 2137 Lab 9: ALU with Input Register}
\author{Megan Gordon}

\maketitle


\section*{Summary}

This lab looks to expand on previous labs by expanding into larger systems with more modules. It allowed us to learn how to create an arithmetic logic unit (ALU) capable of a few operations using hexadecimal values. In completing this lab, the following skills were gained: ability to create and implement a D register with synchronous enable and asynchronous reset, ability to create and implement an arithmetic logic unit (ALU), ability to import and modify modules, and use them to design a modular system. 



\section*{Results}

Below are the simulations with ERTs for 2 modules (register and an ALU) and pictures of the board for each step in the operation list for on-board testing. 

\begin{figure}[ht]\centering
	\begin{tabular}{l|rrrrrrrrrrr}
		Time (ns): & 0-5 & 5-10 & 10-15 & 15-20 & 20-25 & 25-30 & 30-35 & 35-40 & 40-45 & 45-50 & 50-55 \\
		\midrule 
		D (hex) & 0 & 0 & a & a & 3 & 3 & 0 & 0 & 0->6 & 6 & 6  \\
		clk & 0 & 1 & 0 & 1 & 0 & 1 & 0 & 1 & 0 & 1 & 0 \\
		en & 0 & 0 & 1 & 1 & 1->0 & 0->1 & 1->0 & 0 & 0->1 & 1 & 1 \\
		rst & 0 & 0->1 & 0 & 0 & 0 & 0 & 0 & 0 & 0 & 0 & 0 \\
		\midrule
		Q (hex) & X & X->0 & a & a & a & a & a & a & a & 6 & 6  \\
		\bottomrule
	\end{tabular}\medskip

	\includegraphics[width=1.1\textwidth]{register.png}
	\caption{Register ERT and Testbench Results}
	\label{fig:sim_with_table}
\end{figure}


\begin{figure}[ht]\centering
	\begin{tabular}{l|rrrrrr}
		Time (ns): & 0-10 & 10-20 & 20-30 & 30-40 & 40-50 & 50-60 \\
		\midrule
		in0 & 3 & 3 & 3 & 3 & 3 & 3 \\
		in1 & 2 & 2 & 2 & 2 & 2 & 2 \\
		op & 0 & 1 & 2 & 3 & 4 & 7 \\
		\midrule
		out & 5 & 1 & 6 & 3 & 1 & 3 \\
		\bottomrule
	\end{tabular}\medskip
	
	\includegraphics[width=1.1\textwidth]{alu.png}
	\caption{ALU ERT and Testbench Results}
	\label{fig:sim_with_table}
\end{figure}


\clearpage

\begin{figure}[ht]\centering
	\includegraphics[angle=270, width=0.8\textwidth]{step1.jpg}
	\caption{Operation 1-Displaying hexadecimal "14"}
	\label{fig:sim_with_table}
\end{figure}
\clearpage

\begin{figure}[ht]\centering
	\includegraphics[angle=270, width=1.1\textwidth]{step2.jpg}
	\caption{Operation 2-Displaying hexdecimal "14" on LEDs 15-8 and 7-0}
	\label{fig:sim_with_table}
\end{figure}
\clearpage

\begin{figure}[ht]\centering
	\includegraphics[angle=270, width=1.1\textwidth]{step3.jpg}
	\caption{Operation 3-Displaying sum of hexadecimal "14+7A"}
	\label{fig:sim_with_table}
\end{figure}
\clearpage





\section*{Code}

Code for mux2, mux4, andecode, sseg4, sseg4manual is included below.

\begin{lstlisting}[style=Verilog,caption=Mux2 Module Code,label=code:ex ]
`timescale 1ns / 1ps
// Ashlie Lackey and Megan Gordon, ELC 2137, 2020 -03 -05
module mux2 #(parameter N=2)(input [N-1:0]in0, [N-1:0]in1, 
	input sel,
	output [N-1:0]out);
	
	assign out = sel?in1:in0;
endmodule
\end{lstlisting}

\begin{lstlisting}[style=Verilog,caption=Mux4 Module Code,label=code:ex ]
`timescale 1ns / 1ps
// Ashlie Lackey and Megan Gordon, ELC 2137, 2020 -03 -05

module mux4 #(parameter N=4)(input [N-1:0] in3, 
							input [N-1:0] in2, 
							input [N-1:0] in1, 
							input [N-1:0] in0,
							input [1:0] sel,
							output reg [N-1:0] out);
	always @(*)
	begin
		case(sel)
		0: out = in0;
		1: out = in1;
		2: out = in2;
		default: out = in3;
		endcase;
	end
endmodule
\end{lstlisting}

\begin{lstlisting}[style=Verilog,caption=andecoder Module Code,label=code:ex ]
`timescale 1ns / 1ps
// Ashlie Lackey and Megan Gordon, ELC 2137, 2020 -03 -05

module an_decode(input [1:0] in,
	output reg [3:0] out);
	
	always @*
		begin
			case(in)
			0: out = 4'b1110;
			1: out = 4'b1101;
			2: out = 4'b1011;
			default: out = 4'b0111;
			endcase
		end
endmodule
\end{lstlisting}

\begin{lstlisting}[style=Verilog,caption=sseg4 Module Code,label=code:ex ]
`timescale 1ns / 1ps
// Ashlie Lackey and Megan Gordon, ELC 2137, 2020 -03 -05

module sseg4(input [15:0] data,
			input hex_dec,sign,
			input [1:0] digit_sel,
			output reg [7:0] seg,
			output reg dp,
			output reg [3:0] an);

	wire [15:0] bcd11out;
	bcd11 sseg4_bcd11(.B(data[10:0]), .Boutfinal(bcd11out));
	
	wire [15:0] mux2_1_out;
	mux2 #(.N(16))sseg4_mux2_1(.in0(bcd11out), .in1(data[15:0]), .sel(hex_dec), .out(mux2_1_out));
	
	wire [3:0] mux4_out;
	mux4 sseg4_mux4(.in0(mux2_1_out[3:0]), .in1(mux2_1_out[7:4]),.in2(mux2_1_out[11:8]), .in3(mux2_1_out[15:12]), .sel(digit_sel), .out(mux4_out));
	
	wire [6:0]sseg_decoder_out;
	sseg_decoder sseg4_decode(.num(mux4_out), .sseg(sseg_decoder_out));
	
	wire [3:0] decoder_out;
	an_decode an_decode_sseg4(.in(digit_sel), .out(decoder_out));
	
	wire mux22_in;
	assign mux22_in = ~decoder_out[3] & sign;
	mux2 #(.N(7)) sseg4_mux2_2(.in0(sseg_decoder_out), .in1(7'b0111111), .sel(mux22_in), .out(seg));
	
	assign dp = 1;
	assign an = decoder_out;

endmodule
\end{lstlisting}

\begin{lstlisting}[style=Verilog,caption=sseg4 manual Module Code,label=code:ex ]
`timescale 1ns / 1ps
// Ashlie Lackey and Megan Gordon, ELC 2137, 2020 -03 -05

module sseg4_manual(input [15:0]sw,
					output [6:0] seg,
					output dp,
					output [3:0] an);
	
	sseg4 boardconnect(.data({4'b0000, sw[11:0]}), .hex_dec(sw[15]), .sign(sw[14]), .digit_sel(sw[13:12]), .seg(seg), .dp(dp), .an(an));              
endmodule
\end{lstlisting}

\end{document}
